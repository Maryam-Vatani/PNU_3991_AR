\documentclass{article}
\usepackage{multicol}
\usepackage{xcolor}
\usepackage{graphicx}
\linespread{1.35}
\usepackage{amsmath}
\usepackage{color}
\usepackage{tikz}
\usetikzlibrary{arrows,automata}

\begin{document}


\begin{flushright}
 \texttt{Regular Expression} \hspace*{0.1cm}\textbf{$|$} \hspace*{0.1cm} \textbf{269}\hspace*{0.1cm}
\end{flushright}

\vspace*{0.5cm}
\hspace*{0.5cm} The string w cannot be accepted by M, because $\delta(q_0, w)\in F$ [as F does not belong to $(Q - F)$].\\
\hspace*{0.5cm} So, $w \in L$, i.e., $\Sigma * - L \neq L$. But $\Sigma * - L$ is accepted by M' which is an FA.\\
\hspace*{0.5cm} Therefore, $\Sigma^* - L$ is a regular set.\\

\vspace*{0.3cm}
\textbf{Theorem 5.6:} REs are closed under intersection operation.\\

\vspace*{0.1cm}
\textbf{Proof:} From D’Morgan’s theorem, we know\\

\hspace*{4cm} $L_1 \cap L_2 = (L_1^C \cup L_2^C)^C.$

\hspace*{0.5cm} We know that if $L_1$ and $L_2$ are regular, then $L_1^C$ and $L_2^C$ are also regular.\\
\hspace*{0.5cm} As $L_1^C$ and $L_2^C$ are regular, $L_1^C \cup L_2^C$ is also regular (RE is closed under union operation).\\


\hspace*{0.5cm} As $L_1^C \cup L_2^C$ is regular, so complement $(L_1^C \cup L_2^C)C$ is also regular.\\
\hspace*{0.5cm} So,$L_1\cap L_2$ is also regular, i.e., the regular sets are closed under intersection.\\

\vspace*{0.4cm}
\textbf{Theorem 5.7:} Two DFA are closed under cross product.\\

\vspace*{0.3cm}
\textbf{Proof:} Let $D_1 = \{Q_1, \Sigma , \delta_1, q_01, F_1\}$ and $D_2 = \{Q_2, \Sigma , \delta_2, q_02, F_2\}$ are two DFA accepting two RE $L_1$ and
$L_2$ respectively. Let us construct a new FA, D as follows.\\

\vspace*{0.3cm}
\hspace*{4cm} $D = \{Q, \Sigma , \delta , q_0, F\}$ \\

where\\

\hspace*{4cm} $Q = Q_1 \times Q_2$ \\
\small{
\hspace*{0.5cm} $\delta((S_1, S_2), i/p) = (\delta_1(S_1, i/p), \delta_1(S_2, i/p))$ for all $S_1 \in Q_1, S_2 \in Q_2$ and $i/p \in \Sigma.$ }\\


\hspace*{4cm} $q_0 = (q_01, q_02)$ \\
\hspace*{4cm} $F = F_1 \times F_2$ \\

\vspace*{0.3cm}
Clearly D is a DFA. Thus DFA are closed under cross product.\\

\vspace*{0.4cm}
5.13 Decision Problems of Regular Expression\\

\vspace*{0.3cm}
Decision problems are the problems which can be answered in ‘yes’ or ‘no’. FA are one type of finite state
machines which contain a finite number of memory elements. Thus, it can memorize only finite amount
of information. FA can answer to those decision problems which require only a finite amount of memory.\\
\hspace*{0.5cm} The decision problems related to RE are\\

\vspace*{0.3cm}

\begin{enumerate}
  \item Whether a string x belongs to a regular expression $R? (x \in R?)$ \\
  \item Whether the language set of an FA M is empty? $[L(M) = {\O} ?]$ \\
  \item Whether the language set of an FA is finite? [L(M) is finite?] \\
  \item Whether there exists any string accepted by two FA, $M_1$ and $M_2$? \\
  \item Whether the language set of an FA $M_1$ is a subset of the language set of another FA $M_2? [L(M_1) \subseteq
L(M_2)?]$ \\
  \item Whether two FA $M_1$ and $M_2$ accept the same language? $[L(M_1) = L(M_2)]$ \\
  \item Whether two REs $R_1$ and $R_2$ are from the same language set?\\
  \item Whether an FA is the minimum state FA for a language L? \\
\end{enumerate}

\vspace*{0.3cm}
Proof\\
\begin{enumerate}
  \item This problem is known as the membership problem. R can be converted to the equivalent FA M
(see Section 5.5). x is applied on M. If M reaches its final state upon finishing x; $x \in R$ else $x \in R$.\\
\end{enumerate}

\newpage
\begin{flushleft}
    \textbf{270}\hspace*{0.1cm} \textbf{$|$} \hspace*{0.1cm} \texttt{Introduction to Automata Theory, Formal Languages and Computation}
  \end{flushleft}

\vspace*{0.5cm}
\textcolor[rgb]{0.00,0.00,1.00}{2}. An FA does not accept any string if it does not have any fi nal state or if the fi nal state is an
inaccessible state. Let us calculate the set of state $S_k$ reached from the beginning state $q_0$ upon
applying a string of length k.\\

\begin{center}
\section{picture}
\includegraphics[width=8cm,height=1cm]{270.png}
\end{center}

\hspace*{0.3cm} If $k = 0$, it reaches the beginning state $q_0$. It reaches SK if it was in state $S_k-1$ and 'a' as input is
applied on $S_K-1$.\\
\hspace*{0.3cm} Compare the set $S_K$.\\

\vspace*{0.3cm}
\hspace*{0.5cm} a. for string length $k \geq 0$ whether a final state appears or\\
\hspace*{0.5cm} b. $S_K = S_K-1$ (loop) for $k > 0$.\\

\vspace*{0.3cm}
\hspace*{0.5cm} For the case (a), $L(M) \neq {\O}$, and for case (b) $L(M) = {\O}$.\\

\vspace*{0.3cm}

\textcolor[rgb]{0.00,0.00,1.00}{3}. In the pumping lemma (Section 4.11), it is discussed that for accepting a string of length$ \geq n$ (the
number of states of an FA), it has to traverse at least one loop. The language accepted by an FA is
finite if it has length$ < n$. Formulate an algorithm for testing as follows.\\
Give an input to M, the strings of length n or$ > n$ in increasing order. If for a string 's' of length in
between n and $< 2n$, it reaches M then L(M) is infinite; else L(M) is finite.\\
Infinite means there exists x, y, z such that $s = xyz$, where $| xy | \leq n, | y | >0$ and $xy^iz \in L$ for each $i \geq 0$
(pumping lemma). Here, $y^i$ is the looping portion, which can generate an infinite number of strings.\\

\textcolor[rgb]{0.00,0.00,1.00}{4}. For this decidable problem, construct an FA M accepting $L(M_1) \cap L(M_2)$. Then apply the decision
problem 2 on M.\\

\textcolor[rgb]{0.00,0.00,1.00}{5}. For this decidable problem, construct an FA M accepting $L(M_1) - L(M_2)$. Then, apply decision
problem 2 on M. It is true if $L(M_1) - L(M_2) = {\O}$. (If $L(M_1)$ is a subset of $L(M_2)$, $L(M_1) - L(M_2)$
will produce null.)\\


\textcolor[rgb]{0.00,0.00,1.00}{6}. Two sets A and B are the same if $A \subseteq B$ and $B \subseteq A$. Construct an FA, M, accepting $L(M_1) - L(M_2)$
and M' accepting $L(M_2) - L(M_1)$ (reducing it to problem v). Problem vi is decidable if $L(M_1) -
L(M_2) = {\O}$ and $L(M_2) - L(M_1) = {\O}$.\\

\textcolor[rgb]{0.00,0.00,1.00}{7}. There exists an algorithm to convert an RE to FA (see Section 5.5). Reduce this problem to problem (vi).\\

\textcolor[rgb]{0.00,0.00,1.00}{8}. Minimize an FA, M, using 3.13 and generate M'. If the number of states of M and M' are the same,
then it is minimized; else it is not. Hence, decidable.\\

\vspace*{0.2cm}
\large{
\textbf{5.14 ‘Grep’ and Regular Expression}\\
}
\small{
REs have been an integral part of Unix since the beginning. Ken Thompson used the RE in the early
computer text editor ‘QED’ and the Unix editor ‘ed’ for searching text. ‘grep’ is a fi nd string command
in Unix. It is an acronym for ‘globally search a regular expression and print it’. The command searches
through fi les and folders (directories in Unix) and checks which lines in those fi les match a given RE.
‘grep’ gives output the fi lenames and the line numbers or the actual lines that matched the RE.
Let us start with some simple grep commands.}\\

\hspace*{4cm}  \$ grep RE chapter5\\

\hspace*{0.5cm} This command searches the string ‘RE’ in the file ‘chapter5’.\\

\hspace*{4cm}  \$ grep ‘RE’ $-i$ chapter5\\

\newpage
\begin{flushright}
 \texttt{Regular Expression} \hspace*{0.1cm}\textbf{$|$} \hspace*{0.1cm} \textbf{271}\hspace*{0.1cm}
\end{flushright}

\vspace*{0.5cm}
\hspace*{0.5cm} This command searches the strings with case insensitive.\\

\hspace*{4cm}  \$ grep h??a chapter5\\

\hspace*{0.5cm} This command displays all strings of length 4 starting with ‘h’ and ending with ‘a’.\\
\hspace*{0.5cm} If we enter into the internal operation of ‘grep’, we will see that we are searching for a string that
belongs to an RE. Let us take the last grep which searches for four length strings starting with ‘h’ and
ending with ‘a’. For this case, the RE is h$(ch)^+(ch)^+a$, where ‘ch’ is any symbol that belongs to a file
(Generally symbols available in the keyboard).\\

\vspace*{0.3cm}
\large{
\textbf{5.15 Applications of Regular Expression}
}
\small{
RE is mainly used in lexical analysis of compiler design. In the programming language, we need to
declare a variable or identifi er. That identifi er must start with a letter followed by any number of letters
or digits. The structure of the identifi er is represented by RE. The defi nition of an identifi er in a
programming language is\\
}

\hspace*{4cm} letter$ \rightarrow A | B | ... | Z | a | b | ... | z$ \\
\hspace*{4cm} digit $ \rightarrow 0 | 1 | ... | 9$ \\
\hspace*{4cm} id $ \rightarrow letter (letter | digit) *$ \\

\vspace*{0.2cm}
The definition of an unsigned number in a programming language is\\

\vspace*{0.2cm}
\hspace*{4cm} digit $ \rightarrow 0 | 1 | ... | 9$ \\
\hspace*{4cm} digits $ \rightarrow digit +$ \\
\hspace*{3.5cm} opt-fraction $ \rightarrow (. digits) *$ \\
\hspace*{3cm} opt-exponent $ \rightarrow (E (+ | -) * digits) *$ \\
\hspace*{2cm} unsigned-num $\rightarrow$ digits opt-fraction opt-exponent\\

\vspace*{0.2cm}
The other application of RE is pattern searching from a given text.\\

\vspace*{0.4cm}

\begin{center}
\section{picture}
\includegraphics[width=11cm,height=0.4cm]{271.png}
\end{center}


\begin{enumerate}
  \item An RE can be defined as a language or string accepted by an FA.\\
  \item Any terminal symbols, null string $(\wedge)$, or null set $(\Phi)$ are RE.\\
  \item Union, concatenation, iteration of two REs, or any combination of them are also RE.\\
  \item The Arden's theorem states that if P and Q are two REs over $\Sigma$, and if P does not contain $\Lambda$, then
the equation $R = Q + RP$ has a unique (one and only one) solution $R = QP*$.\\
  \item The Arden’s theorem is used to construct an RE from a given FA by the algebraic method.\\
  \item If any FA contains any $\in$ (null) move or transaction, then that FA is called NFA with $\in-moves$.\\
  \item The $\in-closure$ of a state is defined as the set of all states S, such that it can reach from that state to
all the states in S with input $\in$ (i.e., with no input).\\
  \item The pumping lemma for RE is used to prove that certain sets are not regular.\\
  \item A set is closed (under an operation) if and only if the operation on two elements of the set produces
another element of the set.\\
  \item Closure is a property which describes when we combine any two elements of the set, the result is
also included in the set.\\
  \item REs are closed under union, complementation, and intersection operation.\\
\end{enumerate}


\newpage
\begin{flushleft}
    \textbf{272}\hspace*{0.1cm} \textbf{$|$} \hspace*{0.1cm} \texttt{Introduction to Automata Theory, Formal Languages and Computation}
  \end{flushleft}


\begin{center}
\section{picture}
\includegraphics[width=11cm,height=0.5cm]{272.png}
\end{center}


\begin{enumerate}
  \item Describe the following REs in English language.\\
  $a) a(a+b)*abb$   \hspace*{0.4cm}  $b) (a+b)*aba(a+b)*$  \hspace*{0.4cm}  $c) (0+1)*1(0+1)*0(0+1)*$ \\
  \textbf{Solution:}\\
  \begin{itemize}
    \item The language starts with ‘a’ and ends with ‘abb’. In the middle of ‘a’ and ‘b’, there is any
combination of ‘a’ and ‘b’.\\
Hence, the RE described in English language is\\
$\{$Set of any combination of ‘a’ and ‘b’ beginning with ‘a’ and ending with ‘abb’$\}$.\\
    \item The expression is divided into three parts: $(a+b)*$, aba, and $(a+b)*$. In each element of the
language set, there is aba as a substring. In English language, the RE is described as
$\{$Set of any combination of ‘a’ and ‘b’ containing ‘aba’ as substring$\}$.\\
    \item The expression is divided into five parts: $(0+1)*, 1, (0+1)*, 0$ and $(0+1)*$. In each element
of the language set, there is 1 and 0, where 1 appears first. In English language, the RE is
described as\\
$\{$Set of any combination of ‘0’ and ‘1’ containing at least one 1 and one 0 where 1 appears
first$\}$ \\
  \end{itemize}
  \item Find the RE for the following:\\
\end{enumerate}


a) The set of language of all strings of 0 and 1 containing exactly two 0’s\\
\begin{flushright}
   [UPTU 2004]\\
\end{flushright}
b) The set of languages of any combination of ‘a’ and ‘b’ beginning with ‘a’.\\
c) The set of all strings of 0 and 1 that do not end with 11.\\
d) The set of languages of any combination of ‘0’ and ‘1’ containing at least one double symbol.\\
e) The set of all strings over {a, b} in which the number of occurrences of ‘a’ is divisible by 3.\\
\begin{flushright}
  [UPTU 2004]\\
\end{flushright}

f) The set of all strings where the 10th symbol from the right end is a 1.\\
\begin{flushright}
   [JNTU 2007]\\
\end{flushright}
g) The set of languages of any combination of ‘0’ and ‘1’ containing no double symbol.\\
a) The language contains exactly two 0’s. But the language consists of 0 and 1. 1 may appear at
first or in the middle or at the last. Thus, the language is $L = 1*01*01*$.\\
b) The set of any combination of ‘a’ and ‘b’ is denoted by $(a+b)*$. The RE is\\

\vspace*{0.3cm}
\hspace*{4cm} $L = a(a+b)*$. \\

\vspace*{0.3cm}

c) The strings may end with 00, 01, or 10. Thus, the language is $L = (0+1)*(00 + 01+10)$.\\
d) The language consists of two symbols ‘0’ and ‘1’. A double (same) symbol means either 00
or 11. The part of the language containing at least one double symbol is denoted by $(00 + 11)$.
So, the language of any combination of ‘0’ and ‘1’ containing at least one double symbol is
expressed as $L = (0+1)* (00+11) (0+1)*$.\\
e) The number of ‘a’ is divisible by 3 means that the number of ‘a’ may be 0, 3, 6, 9, 12, .....
The number of ‘b’ may be 0, 1, 2, 3,...... . The RE is $(b*ab*ab*ab*)*$.\\
f) The 10th symbol from the right hand side should be ‘a’, whereas the other symbols may be ‘a’
or ‘b’. A string of length n has $(n-10)$ symbols from start and the last 9 symbols are of ‘a’ or
‘b’. The RE is $(a + b)*a$ $(a + b)10$.\\
g) The language consists of two symbols ‘0’ and ‘1’. A double (same) symbol means either 00
or 11. According to the condition, in the language, 00 or 11 will not appear. The language\\


\end{document} 